\newglossaryentry{CSV}
{
  name=CSV-Datei,
  description={\textit{Comma-separated values}\newline Hierbei werden Messungen in einer Textdatei durch Zeilenumbrüche und einzelne Werte durch Beistriche getrennt}
}

\newacronym{cpu}{CPU}{Central Processing Unit}

\newglossaryentry{Ampere}
{
  name=Ampere,
  description={die SI-Basiseinheit der elektrischen Stromstärke}
}

\newglossaryentry{Hertz}
{
  name=Hertz,
  description={die SI-Basiseinheit für die Frequenz\newline Sie gibt die Wiederholungen pro Sekunden an (hier: Schwingungen pro Sekunde}
}

\newglossaryentry{Volt}
{
  name=Volt,
  description={die SI-Basiseinheit der elektrischen Spannung}
}

\newglossaryentry{gpio}
{
  name=GPIO,
  description={General Purpose Input/Output\newline Kontakte, die Softwareseitig für verschiedene Zwecke angesteuert werden können\newline \zB: Auslesen von Sensoren, Ansteuern von Displays}
}

\newglossaryentry{Kernelmodul}
{
  name=Kernelmodul,
  description={ein Programm, welches in das Betriebssystem geladen werden kann und oft zur Unterstützung von Hardware verwendet wird}
}

\newglossaryentry{C}
{
  name=C,
  description={eine sehr weit verbreitete Programmiersprache\newline Hier wird sie oft zum Auslesen der Sensoren verwendet, da sie sehr schnell ausgeführt wird}
}

\newglossaryentry{Bus}
{
  name=Datenbus,
  description={\textit{ein System zur Datenübertragung zwischen mehreren Teilnehmern über einen gemeinsamen Übertragungsweg, bei dem die Teilnehmer nicht an der Datenübertragung zwischen anderen Teilnehmern beteiligt sind.}\footcite{wiki:bus}}
}
\newglossaryentry{I2C}
{
  name=I\textsuperscript{2}C,
  description={\textit{Inter-Integrated Circuit} (auf Deutsch gesprochen: \textit{I-Quadrat-C})\newline ein sehr weit verbreiteter \gls{Bus}}
}

\newacronym{voc}{VOC}{volatile organic compound (dt. Flüchtige organische Verbindungen)}

\newglossaryentry{Javascript}
{
  name=JavaScript,
  description={eine Skriptsprache für dynamische Inhalte in Webseiten}
}

\newglossaryentry{Flickr}
{
  name=Flickr,
  description={ist eine Online-Plattform, auf der Fotos hochgeladen und veröffentlicht werden können}
}

\newglossaryentry{Github}
{
  name=Github,
  description={\emph{ist ein webbasierter Hosting-Dienst für Software-Entwicklungsprojekte}\footcite{wiki:github}}
}
