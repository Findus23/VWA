\newglossaryentry{CSV}
{
  name=CSV-Datei,
  description={\textit{Comma-separated values}\newline Hierbei werden Messungen in einer Textdatei durch Zeilenumbrüche und einzelne Werte durch Beistriche getrennt}
}

\newglossaryentry{CPU}
{
  name=CPU,
  description={\textit{Central Processing Unit}\newline
  	der Hauptprozessor}
}

\newglossaryentry{Ampere}
{
  name=Ampere,
  description={die SI-Basiseinheit der elektrischen Stromstärke}
}

\newglossaryentry{Hertz}
{
  name=Hertz,
  description={die SI-Basiseinheit für die Frequenz\newline Sie gibt die Wiederholungen pro Sekunden an (hier: Wechsel zwischen \emph{Strom} und \emph{kein Strom} in der \gls{CPU} pro Sekunde}
}

\newglossaryentry{Volt}
{
  name=Volt,
  description={die SI-Basiseinheit der elektrischen Spannung}
}

\newglossaryentry{gpio}
{
  name=GPIO,
  description={\emph{General Purpose Input/Output}\newline Kontakte auf der Platine, die softwareseitig für verschiedene Zwecke angesteuert werden können\newline \zB: Auslesen von Sensoren, Ansteuern von Displays}
}

\newglossaryentry{Kernelmodul}
{
  name=Kernelmodul,
  description={ein Programm, welches in das Betriebssystem geladen werden kann und oft zur Kommunikation mit Hardware verwendet wird}
}

\newglossaryentry{1-Wire}
{
  name=1-Wire,
  description={ein \gls{Bus}-System zur einfachen Kommunikation mit Sensoren},
  sort=One-Wire
}

\newglossaryentry{Ohm}
{
  sort=Ohm,
  description={Das Ohm \emph{ist die abgeleitete SI-Einheit des elektrischen Widerstands}\footcite{wiki:ohm}},
  name={\ensuremath{\Omega}}
}

\newglossaryentry{Pascal}
{
  name=Pascal,
  description={ist die Einheit des (Luft-)Drucks}
}


\newglossaryentry{C}
{
  name=C,
  description={eine sehr weit verbreitete Programmiersprache\newline Hier wird sie oft zum Auslesen der Sensoren verwendet, da sie sehr schnell ausgeführt wird}
}

\newglossaryentry{Bus}
{
  name=Datenbus,
  description={\emph{System zur Datenübertragung zwischen mehreren Teilnehmern über einen gemeinsamen Übertragungsweg, bei dem die Teilnehmer nicht an der Datenübertragung zwischen anderen Teilnehmern beteiligt sind.}\footcite{wiki:bus}}
}
\newglossaryentry{I2C}
{
  name=I\textsuperscript{2}C,
  description={\textit{Inter-Integrated Circuit} (auf Deutsch gesprochen: \textit{I-Quadrat-C})\newline ein sehr weit verbreiteter \gls{Bus}}
}

\newacronym{voc}{VOC}{volatile organic compound (dt. Flüchtige organische Verbindungen)}

\newglossaryentry{Javascript}
{
  name=JavaScript,
  description={eine Skriptsprache für dynamische Inhalte in Webseiten}
}

\newglossaryentry{Flickr}
{
  name=Flickr,
  description={ist eine Online-Plattform, auf der Fotos hochgeladen und veröffentlicht werden können}
}

\newglossaryentry{Github}
{
  name=Github,
  description={\emph{ist ein webbasierter Hosting-Dienst für Software-Entwicklungsprojekte}\footcite{wiki:github}}
}

\newglossaryentry{Python}
{
  name=Python,
  description={ist eine 1991 entwickelte Programmiersprache, deren Fokus auf Programmlesbarkeit liegt.\footcite{python}\footcite{python_manual}{}}
}

\newglossaryentry{Bash}
{
  name=Bash,
  description={\textit{Bourne-again shell}\newline ist die heute unter Linux am häufigsten verwendete \gls{Shell}}
}

\newglossaryentry{Shell}
{
  name=Shell,
  description={ist eine Schnittstelle, über die der Benutzer Kommandos an den Computer schicken kann}
}

\newglossaryentry{Einplatinencomputer}
{
  name=Einplatinencomputer,
  description={ein vollständiges Computersystem, welches auf einer einzelnen Platine zusammengefasst ist}
}

\newglossaryentry{a}
{
  name=Einplatinencomputer,
  description={ein vollständiges Computersystem, welches auf einer einzelnen Platine zusammengefasst ist}
}

