\chapter{Auswertung}
\label{cha:Auswertung}

Nun möchte ich die Auswertung der Daten anhand von einer mehrwöchigen Aufzeichnung im Klassenzimmer demonstrieren.

\section{Aufzeichnung}
\label{auswertung_aufzeichnung}

Zu Beginn des Schuljahrs 2014/15 habe ich meine Messtation am 8.9.2014 in der Klasse aufgebaut. Die Messung ging mit nur einigen Minuten Unterbrechung bis zum 3.10.2014. Hierbei wurden ungefähr 2 Mal in der Minute 10 Sensoren ausgelesen. Die dabei entstandene \gls{CSV}, hat 75875 Zeilen und ist knapp über \SI{6}{\mega\byte} groß.\footnote{Sie kann unter \href{http://winkler.kremszeile.at/dygraph_8A.csv}{winkler.kremszeile.at/dygraph\_8A.csv} heruntergeladen werden.}

\section{Graphische Darstellung}

Die Messdaten lassen sich sehr einfach auswerten, wenn man sie als Diagramme darstellt (siehe \ref{subsec:Diagramme}). 

So kann man bei den meisten Sensoren tägliche Schwankungen gut erkennen. Am besten sind sie bei der Luftfeuchtigkeit zu erkennen, welche zwischen ca. \SI{60}{\%} (Mittag bis Abend) und \SI{100}{\%.rel.LF} (Mitternacht bis Vormittag) schwankt. (siehe Abbildung \ref{fig:auswertung-aussen}) 

\begin{figure}[p]
  \centering
     \includegraphics[width=0.95\textheight, angle=90]{figures/auswertung-aussen.png}
  \caption{Außensensoren}
  \label{fig:auswertung-aussen}
\end{figure}

\newpage

\section{Endauswertung}
\label{auswertung_endauswerung}

\begin{tabulary}{\textwidth}{c|c|C|C|C|C}
Sensor & Einheit & Mittelwert & Minimum & Maximum & \gls{Standardabweichung} \\
\hline
Innentemperatur & \si{\degreeCelsius} & 22.44 & 15.375 & 36.437 & 2.97 \\ 
\hline
Gerätetemperatur 1 & \si{\degreeCelsius} & 25.01 & 18.312 & 38.375 & 2.85 \\ 
\hline
Gerätetemperatur 2 & \si{\degreeCelsius} & 25.02 & 18.25 & 38.437 & 2.85 \\ 
\hline
Bodentemperatur & \si{\degreeCelsius} & 14.73 & 4.312 & 44.687 & 3.05 \\ 
\hline
Außentemperatur & \si{\degreeCelsius} & 15.94 & 4.5 & 39.0 & 3.96 \\ 
\hline
Außentemperatur 2 & \si{\degreeCelsius} & 15.87 & 3.5 & 38.7 & 3.95 \\ 
\hline
Luftfeuchtigkeit & \% rel. LF & 82.95 & 14.7 & 99.9 & 12.68 \\ 
\hline
Luftdruck & \si{\hecto\glslink{Pascal}{\pascal}} & 993.26 & 984.04 & 1004.12 & 5.27 \\ 
\hline
Prozessor & \si{\degreeCelsius} & 49.34 & 39.0 & 62.7 & 3.07 \\ 
\hline
Qualität & rel. Wert & 1026.37 & 450.0 & 5870.0 & 529.91 \\ 
\end{tabulary}
