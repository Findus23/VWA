\documentclass[12pt,a4paper,oneside]{scrartcl}
\usepackage[utf8]{inputenc}
\usepackage[ngerman]{babel}
\usepackage[T1]{fontenc}
\usepackage[sc,osf]{mathpazo} 
\usepackage{nag}
\usepackage{graphicx}
\usepackage{caption}
\usepackage[]{enumitem}
\usepackage{multicol}

\setlist{itemsep=-5pt}


%% Einstellungen aus Template
\newcommand{\mycolorlinks}{true}
\newcommand{\myauthor}{Lukas Winkler}
\newcommand{\mytitle}{Begleitprotokoll -- Umweltdatenmessung mit dem Raspberry Pi} 
\newcommand{\mysubject}{\mytitle}
\newcommand{\mykeywords}{Umweltdatenmessung, Raspberry, Temperatur, Klimadaten, Wetter, Auswertung, Software}

\usepackage{xcolor}
\definecolor{DispositionColor}{RGB}{30,103,182}
\input{template/pdf_settings}
\hyphenpenalty=3000 
\tolerance=1000

\author{\myauthor}
\title{Begleitprotokoll}
\subtitle{Umweltdatenmessung mit dem Raspberry Pi}
\begin{document}
\maketitle

\section{Verlauf -- Kontakt mit Betreuungslehrer}
\begin{multicols}{2}
\begin{itemize}

	\item Sommerferien 2013:
	\begin{itemize}
		\item kaufen eines \emph{Raspberry Pi}
		\item experimentieren, testen der Möglichkeiten
	\end{itemize}
	\item September/Oktober 2013:
	\begin{itemize}
		\item eine der ersten Informatik-Stunden: suchen nach Raspberry Pi-Projekten für den Unterricht\newline Idee, eine Wetterstation zu bauen.
		\item bis zur nächsten Stunde: schreiben eines Programmes, welches zufällige Datenreihen erstellt und als Diagramm grafisch darstellt.
	\end{itemize}
	\item 14. Oktober 2013
	\begin{itemize}
		\item einrichten einer Webseite\footnote{\href{http://lukaswiki.onpw.de/rasp/}{lukaswiki.onpw.de/rasp} (nicht mehr erreichbar, Kopie unter \href{http://winkler.kremszeile.at/rasp/}{winkler.kremszeile.at/rasp})}, auf der alle Dateien und Fortschritte protokolliert werden.
	\end{itemize}
	\item 2. November 2013
	\begin{itemize}
		\item erste Teile für die Hardware gekauft (Steckbrett, Verbindungskabel, Temperatursensor)
	\end{itemize}
	\item 14. November 2013
	\begin{itemize}
		\item erste erfolgreiche Messung
	\end{itemize}
	\item 19. November 2013
	\begin{itemize}
		\item erstes Display funktioniert
	\end{itemize}
	\item Dezember 2013
	\begin{itemize}
		\item Messung über zwei Wochen
	\end{itemize}
	\item 20. Dezember 2013
	\begin{itemize}
		\item Treffen mit Betreuungslehrer, Besprechung des aktuellen Zwischenstandes
	\end{itemize}
	\item 27. Dezember 2013
	\begin{itemize}
		\item komplette Projekt ist auf Github\footnote{\href{https://github.com/Findus23/Umweltdatenmessung}{github.com/Findus23/Umweltdatenmessung}} (Versionsverwaltung)
	\end{itemize}
	\item 15. Jänner 2014
	\begin{itemize}
		\item Besprechung der Einreichung mit dem Betreuungslehrer
		\item einreichen der Themenstellung
	\end{itemize}
	\item Jänner 2014
	\begin{itemize}
		\item zusätzliche Sensoren: Luftdruck und Luftfeuchtigkeit
	\end{itemize}
	\item Anfang Februar 2014
	\begin{itemize}
		\item Stabiles Gehäuse für Messtation.
		\item Bericht des Zwischenstandes an den Betreuungslehrer per E-Mail
	\end{itemize}
	\item Ende Februar 2014
	\begin{itemize}
		\item Luftqualitätssensor
	\end{itemize}
	\item April 2014
	\begin{itemize}
		\item Weboberfläche grundlegend verbessert (Anzeige der Live-Werte)
	\end{itemize}
	\item 23. April 2014
	\begin{itemize}
		\item Präsentation bei den \textsf{EDU|days}\footnote{\href{http://www.edudays.at/}{www.edudays.at}}
	\end{itemize}
	\item 17. Juni 2014
	\begin{itemize}
		\item Sieg im Finale vom \textsf{computer creative wettbewerb} des OCG
	\end{itemize}
	\item Sommer 2014
	\begin{itemize}
		\item Artikel im OCG Journal\footnote{\href{http://www.ocg.at/sites/ocg.at/files/medien/pdfs/OCG-Journal1403.pdf}{OCG Journal 3/2014: Seite 33}}
	\end{itemize}
	\item 19. September 2014
	\begin{itemize}
		\item Beginn mit dem Schreiben der VWA
	\end{itemize}
	\item 6. Oktober 2014
	\begin{itemize}
		\item Präsentation beim 3. IKT-Konvent (Arbeitskreis \emph{Bildung, Wissenschaft und Forschung}
	\end{itemize}

\end{itemize}
\end{multicols}
\begin{figure}[h]
  \centering
     \includegraphics[width=\textwidth]{figures/github_verlauf}
  \caption*{Verlauf der Software-Änderungen (Github)}
  \label{fig:github}
\end{figure}

\section{Länge der Arbeit}
\begin{table}[h]
	\centering
	\label{länge}
	\begin{tabular}{c|c|c|c}
	Teil		&	\LaTeX-Code & \LaTeX\ ohne Befehle (detex) & PDF \\
	\hline\hline
	Einleitung	& 1446 & 1311 & 1155\\\hline
	Hardware	 & 11066	 & 8445 & 6793\\\hline
	Software	& 14957 & 12615 & 14493	\\ \hline
	Fazit & 2255 & 2077 & 1898 \\ \hline
	Auswertung & 4017 & 3214 & 2314 \\ \hline\hline
	Gesamt	& 33741  & 27662 & 26653 \\ \hline\hline
	Weitere Informationen & 1190 & 1020 & 812 \\ \hline
	Präsentationen & 2243 & 1940 & 1019 \\ \hline
	Literaturverzeichnis & --- & --- & 5151 \\\hline
	Glosar & 5483 & 4290 & 3336 \\ \hline	\hline
	komplette PDF & --- & --- & 49496 \\
	\end{tabular}
\end{table}
\end{document}
