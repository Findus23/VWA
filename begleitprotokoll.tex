\documentclass[12pt,a4paper,oneside]{scrartcl}
\usepackage[utf8]{inputenc}
\usepackage[ngerman]{babel}
\usepackage[T1]{fontenc}
\usepackage[sc,osf]{mathpazo} 
\usepackage{lipsum}
\usepackage{nag}
%% Einstellungen aus Template
\newcommand{\mycolorlinks}{true}
\newcommand{\myauthor}{Lukas Winkler}
\newcommand{\mytitle}{Begleitprotokoll -- Umweltdatenmessung mit dem Raspberry Pi} 
\newcommand{\mysubject}{\mytitle}
\newcommand{\mykeywords}{Umweltdatenmessung, Raspberry, Temperatur, Klimadaten, Wetter, Auswertung, Software}

\usepackage{xcolor}
\definecolor{DispositionColor}{RGB}{30,103,182}
\input{template/pdf_settings}
\hyphenpenalty=3000 %test von weniger Trennungen
\tolerance=1000

\author{\myauthor}
\title{\mytitle}
\begin{document}
\maketitle

\section{Verlauf -- Kontakt mit Betreuungslehrer}

\begin{itemize}
	\item Sommerferien 2014:
	\begin{itemize}
		\item Ich habe mir einen Raspberry Pi gekauft.
		\item Während der Sommerferien experimentiere ich, was man damit machen kann.
	\end{itemize}
	\item September/Oktober 2014:
	\begin{itemize}
		\item In einer der ersten Informatik-Stunden sollten wir nach Raspberry Pi-Projekten für den Unterricht suchen. So kam ich auf die Idee, eine Wetterstation zu bauen.
		\item Bis zur nächsten Stunde habe ich ein Programm geschrieben, welches zufällige Datenreihen erstellt und als Diagramm grafisch darstellt.
	\end{itemize}
	\item 14. Oktober
	\begin{itemize}
		\item Ich habe eine Webseite\footnote{\href{http://lukaswiki.onpw.de/rasp/}{lukaswiki.onpw.de/rasp/}} eingerichtet, auf der alle Dateien und Fortschritte protokolliert werden.
	\end{itemize}
	\item 2. November
	\begin{itemize}
		\item Erste Teile für die Hardware gekauft (Steckbrett, Verbindungskabel, Temperatursensor)
	\end{itemize}
	\item 14. November
	\begin{itemize}
		\item Erste erfolgreiche Messung
	\end{itemize}

\end{itemize}
\section{Hilfsmittel und Hilfestellungen}
\lipsum[1]
\section{Länge der Arbeit}
\begin{table}[h]
	\centering
	\label{my-label}
	\begin{tabular}{c|c}
	Teil		&	Zeichen \\
		&	\\
		&	\\ \hline
	Gesamt	&  
	\end{tabular}
	\caption{My caption}
\end{table}
\end{document}
