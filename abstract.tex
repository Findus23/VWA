\chapter*{Abstract}
\label{cha:abstract}

Ziel meines Projekt und dieser vorwissenschaftlichen Arbeit darüber ist es, mithilfe eines \emph{Raspberry Pi} Umweltdaten aufzuzeichnen, zu speichern und sowohl grafisch als auch rechnerisch auszuwerten.
Hierzu werden unterschiedlichste Sensoren für Temperatur, Luftdruck, Luftfeuchte und Luftqualität verwendet, welche regelmäßig ausgelesen werden. Die Ergebnisse werden gespeichert und für die spätere Auswertung vorbereitet.

Über ein Interface, welches über den Webbrowser erreichbar ist, werden die aktuellen Messdaten angezeigt und als Balkendiagramm dargestellt. Zusätzlich kann man die komplette Messung als interaktives Diagramm betrachten. Auch ohne einen zusätzlichen Computer zeigen ein Display und drei LEDs den aktuellen Zustand an.

Während einige kleinere Programme, wie ein automatischer Start oder ein tägliches Update, die Bedienung so einfach wie möglich halten, hilft ein Python-Programm bei der mathematischen Auswertung (Berechnung von Mittelwert, Minimum, Maximum und Standardabweichung über einen beliebigen Zeitraum).

Abschließend demonstriere ich die Auswertung anhand einer einmonatigen Messung im Klassenzimmer.