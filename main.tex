%%%% Time-stamp: <2013-09-18 19:27:11 vk>
%% ========================================================================
%%%% Disclaimer
%% ========================================================================
%%
%% created by
%%
%%      Karl Voit
%%

%% ========================================================================
%%%% Basic settings
%% ========================================================================
%% (idea of using newcommands for basic documentclass settings from: Thomas Schlager)

\newcommand{\mypapersize}{A4}
%% e.g., "A4", "letter", "legal", "executive", ...
%% The size of the paper of the resulting PDF file.

\newcommand{\mylaterality}{oneside}
%% "oneside" or "twoside"
%% Either you are creating a document which is printed on both, left pages
%% and right pages (twoside) or you create a document which is printed
%% on right pages only (oneside).

\newcommand{\mydraft}{false}
%% "true" or "false"
%% Use draft mode? If true, included graphics are replaced by empty
%% rectangles (of same size) and overfull boxes (in margin space) are
%% marked with black box (-> easy to spot!)

\newcommand{\myparskip}{half}
%% e.g., "no", "full", "half", ...
%% How to separate paragraphs: indention ("no") or spacing ("half",
%% "full", ...).

\newcommand{\myBCOR}{0mm}
%% Inner binding correction. This value depends on the method which is
%% being used to bind your printed result. Some techniques do not
%% require a binding correction at all ("0mm"), other require for
%% example "5mm". Refer to KOMA script documentation for a detailed
%% explanation what a binding correction is and how to measure it.

\newcommand{\myfontsize}{12pt}   
%% e.g., 10pt, 11pt, 12pt
%% The font size of the main text in pt (points).

\newcommand{\mylinespread}{1.0} 
%% e.g., 1.0, 1.5, 2.0
%% Line spacing in %/100. For example 1.5 means 150% of the usual line
%% spacing. Please use with caution: 100% ("1.0") is fine because the
%% font was designed for it.

\newcommand{\mylanguage}{american,ngerman}
%% "english,ngerman", "ngerman,english", ...
%% NOTE: The *last* language is the active one!
%% See babel documentation for further details.

%% BibLaTeX-settings: (see biblatex reference for further description)
\newcommand{\mybiblatexstyle}{authoryear-ibid}
%% e.g., "alphabetic", "authoryear", ...
%% The biblatex style which is being used for referencing. See
%% biblatex documentation for further details and more values.
%%
%% CAUTION: if you change the style, please check for (in)compatible
%%          "biblatex" package options in the file
%%          "template/preamble.tex"! For example: "alphabetic" does
%%          not have an option "dashed=..." and causes an error if it
%%          does not get removed from the list of options.

\newcommand{\mybiblatexdashed}{true}  %% "true" or "false"
%% If true: replace recurring reference authors with a dash.

\newcommand{\mybiblatexbackref}{true}  %% "true" or "false"
%% If true: create backward links from reference to citations.

\newcommand{\mybiblatexfile}{references-biblatex.bib}
%% Name of the biblatex file that holds the references.

\newcommand{\mydispositioncolor}{30,103,182}
%% e.g., "30,103,182" (blue/turquois), "0,0,0" (black), ...
%% Color of the headings and so forth in RGB (red,green,blue) values.
%% NOTE: if you are using "0,0,0" for black, printers might still
%%       recognize pages as color pages. In case this is a problem
%%       (paying for color print-outs vs. paying for b/w-printouts)
%%       please edit file "template/preamble.tex" and change
%%       "\definecolor{DispositionColor}{RGB}{\mydispositioncolor}"
%%       to "\definecolor{DispositionColor}{gray}{0}" and thus
%%       overwriting the value of \mydispositioncolor above.

\newcommand{\mycolorlinks}{true}  %% "true" or "false"
%% Enables or disables colored links (hyperref package).

\newcommand{\mytitlepage}{template/title_VWA}
%% Your own or one of following pre-defined title pages:
%% "template/title_plain_maketitle": simple maketitle page
%% "template/title_Diplomarbeit_KF_Uni_Graz.tex": fancy (german) title page for KF Uni Graz
%% "template/title_Thesis_TU_Graz": titlepage for Graz University of Technology (correct Corporate Design)
%% "template/title_VWA": titlepage for Vorwissenschaftliche Arbeit

\newcommand{\mytodonotesoptions}{}
%% e.g., "" (empty), "disable", ...
%% Options for the todonotes-package. If "disable", all todonotes will
%% be hidden (including listoftodos).

%% Load main settings for document preamble:
\input{template/preamble}%% DO NOT REMOVE THIS LINE!

\setboolean{myaddcolophon}{true}  %% "true" or "false"
%% If set to "true": a colophon (with notes about this document
%% template, LaTeX, ...) is added after the title page.
%% Please do not set to "false" without a good reason. The colophon
%% helps your readers to get in touch with LaTeX and to find this template.

\setboolean{myaddlistoftodos}{true}  %% "true" or "false"
%% If set to "true": the current list of open todos is added after the
%% table of contents. If \mytodonotesoptions is set to "disable", no
%% list of todos is added, independent of this setting here.



%% ========================================================================
%%%% Document metadata
%% ========================================================================

%% general metadata:
\newcommand{\myauthor}{Lukas Winkler}  %% also used for PDF metadata (hyperref)
\newcommand{\mytitle}{Umweldatenmessung mit dem Raspberry Pi}  %% also used for PDF metadata (hyperref)
\newcommand{\mysubject}{Umweldatenmessung mit dem Raspberry Pi}  %% also used for PDF metadata (hyperref)
\newcommand{\mykeywords}{Umweldatenmessung, Raspberry, Temperatur, Klimadaten, Wetter, Auswertung, Software}  %% also used for PDF metadata (hyperref)

%% this information is used only for generating the title page:
\newcommand{\myworktitle}{Master's Thesis}  %% official type of work like ``Master theses''
\newcommand{\mygrade}{Master of Science} %% title you are getting with this work like ``Master of ...''
\newcommand{\mystudy}{8A} %% your study like ``Arts''
\newcommand{\myuniversity}{Graz University of Technology} %% your university/school
\newcommand{\myinstitute}{BRG Rechte Kremszeile} %% affiliation
\newcommand{\myinstitutehead}{Univ.-Prof.\,Dipl-Ing.\,Dr.techn.~Some One} %% head of institute
\newcommand{\mysupervisor}{MMag. Matthias Kittel} %% your supervisor
\newcommand{\myevaluator}{Prof.~Some Genius} %% your evaluator
\newcommand{\myhomestreet}{Rechte Kremszeile 54} %% your home street (with house number)
\newcommand{\myhometown}{Krems an der Donau} %% your home town
\newcommand{\myhomepostalnumber}{3500} %% your postal number of home town
\newcommand{\mysubmissionmonth}{Januar} %% month you are handing in
\newcommand{\mysubmissionyear}{2015} %% year you are handing in
\newcommand{\mysubmissiontown}{\myhometown} %% town of handing in (or \myhometown)

%% additional information for generic_documentation title page
\newcommand{\myid}{1234567} %% Matrikelnummer
\newcommand{\mylecture}{LECTURE} %% 


%% ========================================================================
%%%% MISC command definitions
%% ========================================================================
\input{template/mycommands}

%% ========================================================================
%%%% Typographic settings
%% ========================================================================
\input{template/typographic_settings}


%% ========================================================================
%%%% MISC usepackages
%% ========================================================================

%% ... it's OK to put here your own usepackage commands ...
\usepackage{listings}
\definecolor{mygreen}{rgb}{0,0.6,0}
\definecolor{mygray}{rgb}{0.5,0.5,0.5}
\definecolor{mymauve}{rgb}{0.58,0,0.82}

\lstdefinestyle{mystyle}{
	numberbychapter=false,
	backgroundcolor=\color{white},
%	basicstyle=\footnotesize,
	breakatwhitespace=false,
	breaklines=true,
	captionpos=b,
	commentstyle=\color{mygreen},
	frame=shadowbox,
	keepspaces=true,
	keywordstyle=\color{blue},
	numbers=left,
	numbersep=5pt,
	numberstyle=\color{mygray},
	rulecolor=\color{black},
	rulesepcolor=\color{gray},
	showspaces=false,
	showstringspaces=true,
	showtabs=true,
	stepnumber=1,
	stringstyle=\color{mymauve},
	tabsize=2,
}
\lstset{literate=
  {á}{{\'a}}1 {é}{{\'e}}1 {í}{{\'i}}1 {ó}{{\'o}}1 {ú}{{\'u}}1
  {Á}{{\'A}}1 {É}{{\'E}}1 {Í}{{\'I}}1 {Ó}{{\'O}}1 {Ú}{{\'U}}1
  {à}{{\`a}}1 {è}{{\`e}}1 {ì}{{\`i}}1 {ò}{{\`o}}1 {ù}{{\`u}}1
  {À}{{\`A}}1 {È}{{\'E}}1 {Ì}{{\`I}}1 {Ò}{{\`O}}1 {Ù}{{\`U}}1
  {ä}{{\"a}}1 {ë}{{\"e}}1 {ï}{{\"i}}1 {ö}{{\"o}}1 {ü}{{\"u}}1
  {Ä}{{\"A}}1 {Ë}{{\"E}}1 {Ï}{{\"I}}1 {Ö}{{\"O}}1 {Ü}{{\"U}}1
  {â}{{\^a}}1 {ê}{{\^e}}1 {î}{{\^i}}1 {ô}{{\^o}}1 {û}{{\^u}}1
  {Â}{{\^A}}1 {Ê}{{\^E}}1 {Î}{{\^I}}1 {Ô}{{\^O}}1 {Û}{{\^U}}1
  {œ}{{\oe}}1 {Œ}{{\OE}}1 {æ}{{\ae}}1 {Æ}{{\AE}}1 {ß}{{\ss}}1
  {ç}{{\c c}}1 {Ç}{{\c C}}1 {ø}{{\o}}1 {å}{{\r a}}1 {Å}{{\r A}}1
  {€}{{\EUR}}1 {£}{{\pounds}}1 {}{}0,
	style=mystyle,
}
\renewcommand{\lstlistingname}{Programmcode}
\renewcommand{\lstlistlistingname}{Programmcode}
\newcommand{\code}[4]{
	\lstinputlisting[language=#2,firstline=#3,lastline=#4,caption=#1 (Zeile #3 bis #4),firstnumber=#3]{code/#1}
}
\newcommand{\codeline}[3]{
	\lstinputlisting[language=#2,firstline=#3,lastline=#3,caption=#1 (Zeile #3),firstnumber=#3]{code/#1}
}
\usepackage[binary-units = true]{siunitx}
\usepackage{wrapfig} % Bildumlauf
\usepackage[toc,nopostdot]{glossaries}
\glsenablehyper
\newglossaryentry{CSV}
{
  name=CSV-Datei,
  description={\textit{Comma-separated values}\newline Hierbei werden Messungen in einer Textdatei durch Zeilenumbrüche und einzelne Werte durch Beistriche getrennt}
}

\newglossaryentry{CPU}
{
  name=CPU,
  description={\textit{Central Processing Unit}\newline
  	der Hauptprozessor}
}

\newglossaryentry{Ampere}
{
  name=Ampere,
  description={die SI-Basiseinheit der elektrischen Stromstärke}
}

\newglossaryentry{Hertz}
{
  name=Hertz,
  description={die SI-Basiseinheit für die Frequenz\newline Sie gibt die Wiederholungen pro Sekunden an (hier: Wechsel zwischen \emph{Strom} und \emph{kein Strom} in der \gls{CPU} pro Sekunde}
}

\newglossaryentry{Volt}
{
  name=Volt,
  description={die SI-Basiseinheit der elektrischen Spannung}
}

\newglossaryentry{gpio}
{
  name=GPIO,
  description={\emph{General Purpose Input/Output}\newline Kontakte auf der \gls{Platine}, die softwareseitig für verschiedene Zwecke angesteuert werden können\newline \zB: Auslesen von Sensoren, Ansteuern von Displays}
}

\newglossaryentry{Kernelmodul}
{
  name=Kernelmodul,
  description={ein Programm, welches in das Betriebssystem geladen werden kann und oft zur Kommunikation mit Hardware verwendet wird}
}

\newglossaryentry{1-Wire}
{
  name=1-Wire,
  description={ein \gls{Bus}-System zur einfachen Kommunikation mit Sensoren},
  sort=One-Wire
}

\newglossaryentry{Ohm}
{
  sort=Ohm,
  description={Das Ohm \emph{ist die abgeleitete SI-Einheit des elektrischen Widerstands}\footcite{wiki:ohm}},
  name={\ensuremath{\Omega}}
}

\newglossaryentry{Pascal}
{
  name=Pascal,
  description={die Einheit des (Luft-)Drucks}
}


\newglossaryentry{C}
{
  name=C,
  description={eine sehr weit verbreitete Programmiersprache\newline Hier wird sie oft zum Auslesen der Sensoren verwendet, da sie sehr schnell ausgeführt wird}
}

\newglossaryentry{Bus}
{
  name=Datenbus,
  description={\emph{System zur Datenübertragung zwischen mehreren Teilnehmern über einen gemeinsamen Übertragungsweg, bei dem die Teilnehmer nicht an der Datenübertragung zwischen anderen Teilnehmern beteiligt sind.}\footcite{wiki:bus}}
}
\newglossaryentry{I2C}
{
  name=I\textsuperscript{2}C,
  description={\textit{Inter-Integrated Circuit} (auf Deutsch gesprochen: \textit{I-Quadrat-C})\newline ein sehr weit verbreiteter \gls{Bus}}
}

\newacronym{voc}{VOC}{volatile organic compound (dt. Flüchtige organische Verbindungen)}

\newglossaryentry{Javascript}
{
  name=JavaScript,
  description={eine Skriptsprache für dynamische Inhalte in Webseiten}
}

\newglossaryentry{Flickr}
{
  name=Flickr,
  description={eine Online-Plattform, auf der Fotos hochgeladen und veröffentlicht werden können}
}

\newglossaryentry{Github}
{
  name=Github,
  description={\emph{ein webbasierter Hosting-Dienst für Software-Entwicklungsprojekte}\footcite{wiki:github}}
}

\newglossaryentry{Python}
{
  name=Python,
  description={eine 1991 entwickelte Programmiersprache, deren Fokus auf Programmlesbarkeit liegt.\footcite{python}$^,$\footcite{python_manual}{}}
}

\newglossaryentry{Bash}
{
  name=Bash,
  description={\textit{Bourne-again shell}\newline die heute unter Linux am häufigsten verwendete \gls{Shell}}
}

\newglossaryentry{Shell}
{
  name=Shell,
  description={eine Schnittstelle, über die der Benutzer Kommandos an den Computer schicken kann}
}

\newglossaryentry{Einplatinencomputer}
{
  name=Einplatinencomputer,
  description={ein vollständiges Computersystem, welches auf einer einzelnen \gls{Platine} zusammengefasst ist}
}

\newglossaryentry{LC-Display}
{
  name=LC-Display,
  description={\emph{liquid crystal display}\newline Flüssigkristallbildschirm}
}

\newglossaryentry{Platine}
{
  name=Platine,
  description={auch genannt Leiterplatte\newline
  ein Träger für elektrische Bauteile\newline
  \includegraphics[width=4cm]{figures/platine.png}\footcite{platine}
    	}
}

\newglossaryentry{Streifenplatine}
{
  name=Streifenplatine,
  plural=Streifenplatinen,
  description={eine \gls{Platine}, bei der die Kontakte streifenförmig miteinander verbunden sind.\newline
  \includegraphics[width=4cm]{figures/streifenplatine.png}\footcite{streifenplatine}
  	}
}

\newglossaryentry{LED}
{
  name=LED,
  description={\emph{light-emitting diode}\newline Licht abgebende Diode}
}

\newglossaryentry{Geraetedatei}
{
  name=Gerätedatei,
  description={eines der grundlegenden Prinzipien von diversen Linux-Betriebssystemen ist \emph{Everything is a file}.\newline
  Daher können auf Festplatten, Schnittstellen und Informationen über das System einfach über das Auslesen von Dateien zugegriffen werden.
  }
}

\newglossaryentry{Standardabweichung}
{
  name=Standardabweichung,
  description={ein Maß für die Streuung von Werten}
}

\newglossaryentry{Steckbrett}
{
  name=Steckbrett,
  description={hierauf können schnell Schaltungen aufgebaut und getestet werden}
}

\newglossaryentry{Linux-Distribution}
{
  name=Linux-Distribution,
  plural=Linux-Distributionen,
  description={Es gibt nicht nur ein \emph{Linux}, sondern eine sehr große Menge\footnote{siehe diese Grafik: \href{http://de.wikipedia.org/wiki/Datei:Linux_Distribution_Timeline.svg}{de.wikipedia.org/wiki/Datei:Linux\_Distribution\_Timeline.svg}} Betriebssysteme, welche alle auf dem ursprünglichen \emph{Linux}-Kernel basieren.}
}

\newglossaryentry{Gnuplot}
{
  name=Gnuplot,
  description={ein Programm \emph{zur grafischen Darstellung von Messdaten und mathematischen Funktionen}\footcite{wiki:gnuplot}}
}
\makeglossaries

\hyphenpenalty=3000 %test von weniger Trennungen
\tolerance=1000


%% ========================================================================
%%%% MISC self-defined commands and settings
%% ========================================================================

%% ... it's OK to put here your own newcommand/newenvironment-definitions ...




\newcommand{\myLaT}{\LaTeX{}@TUG\xspace} %% LaTeX@TUG text "logo"

\hyphenation{ex-am-ple hy-phen-ate Gehäuse-temperatur Innen-temperatur Temperatur-sensoren}  %% in order to use German umlauts
%% here (Ver-\"of-fent-li-chung), you have to check for 
%% activated \usepackage[T1]{fontenc} in the preamble

%% override default language of babel: (be sure to know, what you're
%% doing here)
%\selectlanguage{american}
%\selectlanguage{ngerman}

%% ========================================================================
%%%% Templates
%% ========================================================================

%% template for inserting figures:
% \myfig{}%% filename
%       {}%% width/height
%       {}%% caption
%       {}%% optional (short) caption for list of figures
%       {fig:}%% label

%% acronyms in small caps: \myacro{UNESCO}


\input{template/pdf_settings}  %% should be *last* definitions in preamble!
%% ========================================================================
%%%% begin{document}
%% ========================================================================
\begin{document}

\frontmatter                    %% KOMA: roman page numbers and such; only available in scrbook

\input{colophon}                %% defines information about editor, LaTeX, font, ...

%% Choose your desired title page:
\input{\mytitlepage}            %% include title page


%% include the abstract without chapter number but include it on table of contents:
\cleardoublepage
\addcontentsline{toc}{chapter}{Abstract}
\chapter*{Abstract}
\label{cha:abstract}

Ziel meines Projekt und dieser vorwissenschaftlichen Arbeit darüber ist es, mithilfe eines \emph{Raspberry Pi} Umweltdaten aufzuzeichnen, zu speichern und sowohl grafisch als auch rechnerisch auszuwerten.
Hierzu werden unterschiedlichste Sensoren für Temperatur, Luftdruck, Luftfeuchte und Luftqualität verwendet, welche regelmäßig ausgelesen werden. Die Ergebnisse werden gespeichert und für die spätere Auswertung vorbereitet.

Über ein Interface, welches über den Webbrowser erreichbar ist, werden die aktuellen Messdaten angezeigt und als Balkendiagramm dargestellt. Zusätzlich kann man die komplette Messung als interaktives Diagramm betrachten. Auch ohne einen zusätzlichen Computer zeigen ein Display und drei LEDs den aktuellen Zustand an.

Während einige kleinere Programme, wie ein automatischer Start oder ein tägliches Update, die Bedienung so einfach wie möglich halten, hilft ein Python-Programm bei der mathematischen Auswertung (Berechnung von Mittelwert, Minimum, Maximum und Standardabweichung über einen beliebigen Zeitraum).

Abschließend demonstriere ich die Auswertung anhand einer einmonatigen Messung im Klassenzimmer.              %% Abstract


\tableofcontents                %% this produces the table of contents - you might have guessed :-)


%% if myaddlistoftodos is set to "true", the current list of open todos is added:
\ifthenelse{\boolean{myaddlistoftodos}}{
  \newpage\listoftodos          %% handy if you are using todonotes with \todo{}
}{}                             %% with todonotes-package option "disable" you can get rid of any todo in the output

\mainmatter                     %% KOMA: marks main part using arabic page numbers and such; only available in scrbook
% Glossareinträge
\newglossaryentry{CSV}
{
  name=CSV-Datei,
  description={\textit{Comma-separated values}\newline Hierbei werden Messungen in einer Textdatei durch Zeilenumbrüche und einzelne Werte durch Beistriche getrennt
  \begin{verbatim}
  23,21,75
  25,20,73
  25,19,75
  \end{verbatim}
  }
}
% Ende Glossareinträge
\chapter{Einleitung}

Im letzten Jahr habe ich mich damit beschäftigt, wie man mithilfe eines Raspberry Pi Umweltdaten messen, aufzeichnen und auswerten kann. Hierzu verwende ich mehrere Sensoren, die Lufttemperatur (sowohl im Klassenraum, als auch außen), Luftfeuchtigkeit, Luftdruck und die relative Luftqualität. Diese Daten werden als \gls{CSV} gespeichert und können grafisch und rechnerisch ausgewertet werden.  

\chapter{Hardware}

Die Hardware besteht aus einem Raspberry Pi, \todo{genauere Beschreibung}
\section{Der Raspberry Pi}
\begin{wrapfigure}{r}{0.5\textwidth}
 \vspace{-16pt}
 \centering 
 \includegraphics[width=0.45\textwidth]{figures/raspberry.jpg}
 \caption[Raspberry Pi - Modell B]{Raspberry Pi - Modell B\footnotemark}
 \vspace{-50pt}
\end{wrapfigure}
\footnotetext{\cite{rasp_bild}}
Der \textit{Raspberry Pi} ist ein Einplatinencomputer, der 2012 von der \textit{Raspberry Pi Foundation} auf den Markt gebracht wurde.
\subsection{Geschichte}
Ursprünglich war er als günstiger Computer gedacht, um britischen Jugendlichen das Programmieren näher zu bringen. An der \textit{University of Cambridge} stellte man fest, dass die Vorkenntnisse von Studienanfängern immer geringer wurden, weil sie -- sowohl privat als auch in der Schule -- sich immer weniger mit der Funktionsweise von Computern und Programmen beschäftigen. Daher wollte man einen Computer entwickeln, mit dem die Jugendlichen experimentieren können.
\footcite{aboutraspberry}$^,$\todo{absolut geschummelt}
\footcite[Geschichte]{wiki:raspberry}

\subsection{Technische Daten}
Die Technik in einem Raspberry Pi ist vergleichbar mit der eines Smartphones. Der Raspberry Pi hat eine \acrshort{cpu} mit 700 MHz, welche auf bis zu 1 GHz übertaktbar ist, und je nach Modell 256 oder 512 MB Arbeitsspeicher. Als Speichermedium für das Betriebssystem (verschiedene Linux-Distributionen stehen zur Auswahl) wird eine SD-Karte bzw. eine microSD-Karte verwendet.

Zur Stromversorgung genügt ein normales Handy-Ladegerät mit Micro-USB-Anschluss und 1 \gls{Ampere} Stromstärke, denn der Raspberry Pi benötigt nur 3,5 Watt\footcite{strom} (Modell B).

Zum Anschließen anderer Hardware gibt es zwei USB-Anschlüsse und 26 \gls{gpio}-Pins.

\section{Sensoren}
\label{sec:Sensoren}

Zur Messung der Werte werden folgende Sensoren verwendet:
\begin{itemize}
\item 4 Temperatursensoren \textit{DS18B20}
\item Luftfeuchtesensor \textit{DHT22}
\item Luftdrucksensor \textit{BMP085}
\item Luftqualitätssensor \textit{VOLTCRAFT CO-20}
\item CPU-Temperatur des Raspberry Pi
\end{itemize}
\subsection{Temperatur}
\label{subsec:Temperatur}

%\begin{figure}
%  \centering
%     \includegraphics[width=0.4\textwidth]{figures/temp_sensor.jpg}
%  \caption{Platine mit dem DS18B20 für den Innensensor (eigenes Werk)}
%  \label{fig:temp_sensor}
%\end{figure}


Mithilfe der Temperatur"-sensoren werden die Innen"-temperatur, die Gehäuse"-temperatur und die Bodentemperatur (Außen) gemessen. Der hat eine Messgenauigkeit von \SI{\pm 0.5}{\degreeCelsius}  und einen Messbereich von \SI{-10}{\degreeCelsius}  bis \SI{+85}{\degreeCelsius}. \footcite[20]{temp}

Der Sensor wird mithilfe von einem 1-Wire-Bus ausgelesen. Hierbei benötigt man (außer für die Stromversorgung mit 5 \gls{Volt}) nur ein Kabel, auf dem die Daten übertragen werden.\footcite{1-wire}
Ein weiterer Vorteil von 1-Wire ist, dass nahezu beliebig viele Sensoren parallel geschaltet werden können. (Abb. \ref{fig:temp_pin})

Die Messdaten des \textit{DS18B20} können auf dem Raspberry Pi sehr einfach ausgelesen werden, weil dies von einem Linux-\gls{Kernelmodul} erledigt wird. Um die Temperatur zu erhalten, muss  nur eine virtuelle Datei auslesen werden, welche das Messergebnis in tausendstel Grad Celsius enthält. (Siehe Abbildung \ref{fig:temp_screenshot})
\begin{figure}
  \centering
     \includegraphics[width=0.4\textwidth]{figures/temp_pin.png}
  \caption{Pinbelegung des DS18B20 (eigenes Werk)}
  \label{fig:temp_pin}
\end{figure}
\begin{figure}[h]
  \centering
     \includegraphics[width=\textwidth]{figures/temp_screenshot.png}
  \caption{Die erste erfolgreiche Messung (eigenes Werk)}
  \label{fig:temp_screenshot}
\end{figure}

\subsection{Luftfeuchtigkeit}
\label{subsec:Luftfeuchtigkeit}

Zum Messen der Luftfeuchtigkeit der Außenluft wird der \textit{DHT22} verwendet. Dieser kann auch die Temperatur messen. 
Wie der \textit{DS18B20} (\ref{subsec:Temperatur}) benötigt der Luftfeuchtigkeitssensor zusätzlich zur Stromversorgung nur ein Kabel zur Datenübertragung. Es können jedoch nicht mehrere Sensoren parallel geschaltet werden. \footcite[Wiring]{DHT}

\begin{figure}[h]
  \centering
     \includegraphics[width=\textwidth]{figures/steckbrett.png}
  \caption{Anschlussskitze von \textit{DS18B20} (\ref{subsec:Temperatur}; Mitte), \textit{DHT22} (\ref{subsec:Luftfeuchtigkeit}; Links) und \textit{BMP085} (\ref{subsec:Luftdruck}; Rechts) (eigenes Werk)}
  \label{fig:steckbrett}
\end{figure}

Die Daten des Sensors werden von einem \gls{C} Programm von Adafruit ausgelesen.
\footcite[Software Install]{DHT}

\subsection{Luftdruck}
\label{subsec:Luftdruck}

Der \textit{BMP085} ist der präziseste Sensor. Er wird zum Messen des Luftdruckes und der Außentemperatur verwendet und hat dabei eine Genauigkeit von\SI{\pm 1.0}{\hecto\pascal} und \SI{0.5}{\degreeCelsius} bei \SI{25}{\degreeCelsius}\footcite[6]{BMP085}

Die Messdaten überträgt der Sensor über einen \gls{I2C}-Bus. Dabei werden (zusätzlich zur Stromversorgung) \textbf{zwei} Kabel zur Stromversorgung benötigt. (siehe Abbildung \ref{fig:steckbrett})
Zum einen ist das das gelbe Kabel, über welches der Raspberry Pi dem Sensor die Taktfrequenz schickt, in dem er die Daten übertragen soll, und das grüne Kabel, über das die eigentlichen Daten übertragen werden.
\footcite[Hooking Everything Up]{bmp058_adafruit}

Auch dies wird von einem Programm von Adafruit übernommen. \footcite[Using the Adafruit BMP Python Library (Updated)]{bmp058_adafruit}

\subsection{Luftqualität}
\label{subsec:Luftqualitat}

\chapter{Software}
Die Software, die verwendet wird, teilt sich in (?) Teile auf:
\begin{itemize}
\item Auslesen der Sensoren, Aufbereiten der Daten und allgemeine Steuerung (main.sh)
\item Steuern des Displays
\item Endauswertung
\item Webinterface
\item sonstiges
\end{itemize}

\section{main.sh}
\label{sec:main.sh}

Das wichtigste Programm ist das Bash-Script \textit{main.sh}. Mithilfe eines Bash-Scriptes können Programme automatisiert gestartet und ihre Ausgaben ausgewertet werden.
Die Datei beginnt mit einem \textit{Shebang} (auch \textit{Magic Line}) genannt. Diese Zeile sagt dem Betriebssystem, womit die Datei ausgeführt werden soll.
\codeline{main.sh}{bash}{1}
Die folgenden Zeilen geben allgemeine Einstellungen an und definieren später gebrauchte Variablen. Man kann den Pfad zum Webserver, auf dem das Webinterface liegt, angeben. In Zeile 6 und 7 werden die Zugangsdaten für Pushbullet aus einer anderen Datei ausgelesen. Die Zeilen 8-10 geben die Pins an, an denen die LEDs angeschlossen sind. In Zeile 11-13 wird die grüne LED eingeschaltet, um zu zeigen, dass das Programm läuft.
\code{main.sh}{bash}{2}{13}
Zeile 14 bis 26 kümmern sich um Parameter, die an das Programm übergeben werden. Mit \enquote{-d} kann die letzte Aufzeichnung überschrieben werden und mit \enquote{-h} wird ein kurzer Info-Text angezeigt.
 
\begin{lstlisting}[language=bash,style=terminal]
lukas@kinderzimmer:~$ main.sh -h
-d csv-Datei leeren 
für weitere Informationen siehe http://winkler.kremszeile.at/ oder https://github.com/Findus23/Umweltdatenmessung

\end{lstlisting}
\code{main.sh}{bash}{14}{26}
Alles, was nun folgt, wird unendlich wiederholt, bis das Programm beendet wird. 
\code{main.sh}{bash}{27}{28}
In den folgenden drei Zeilen wird die aktuelle Uhrzeit und Datum in drei verschiedenen Formaten für drei verschiedene Zwecke.
\begin{table}[h]
	\centering
	\begin{tabulary}{\textwidth}{C|C|C}
		Code & Beispiel & Verwendung \\
		\hline
		\hline
		\%Y/\%m/\%d\ \%H:\%M:\%S & 2014/11/23 16:47:50 & Format zum Abspeichern in \gls{CSV} \\
		\hline
		\%d.\%m\ \%H:\%M:\%S & 23.11 16:47:50 & einfach lesbares Format für Display \\
		\hline
		\%d.\%m.\%y\ \%H:\%M:\%S & 23.11.2014 & einfaches, exaktes Format für Webinterface \\
	\end{tabulary}
	\caption{Datumsformate}
\end{table}


\code{main.sh}{bash}{29}{31}
\codeline{main.sh}{bash}{32}


\appendix                       %% closes main document, appendix follows until end; only available in book-classes
\addpart*{Anhang}             %% adding Appendix to tableofcontents

\printbibheading
\printbibliography[type=book,heading=subbibliography,title={Literatur}]
\printbibliography[type=online,heading=subbibliography,title={Online-Literatur}]
\printbibliography[nottype=online,nottype=book,heading=subbibliography,title={sonstige Literatur}]

\listoffigures
%\listoftables
\lstlistoflistings
\glsaddall
\setglossarystyle{altlisthypergroup} % alternativ listhypergroup (in der selben Zeile)
\printglossary[title=Glossar]

\chapter*{Eidesstattliche Erklärung}
Ich, \myauthor, erkläre hiermit eidesstattlich, dass ich diese
vorwissenschaftliche Arbeit selbständig und ohne Hilfe Dritter
verfasst habe.  Insbesondere versichere ich, dass ich alle wörtlichen
und sinngemäßen Übernahmen aus anderen Werken als Zitate kenntlich
gemacht und alle verwendeten Quellen angegeben habe.

\vfill
\newcommand{\mysignatureblock}[3]{%
  \begin{tabular}{llp{2em}l} 
  #1 & \hspace{3cm}        & & \hspace{4cm} \\\cline{2-2}\cline{4-4}
     &                     & & \\[-3mm]
     & {\footnotesize #2}  & & {\footnotesize #3}
  \end{tabular}
}
\mysignatureblock{\myhometown, am}{Datum}{Unterschrift}
\vfill\vfill




%%%% end{document}
\end{document}
%% vim:foldmethod=expr
%% vim:fde=getline(v\:lnum)=~'^%%%%\ .\\+'?'>1'\:'='
%%% Local Variables:
%%% mode: latex
%%% mode: auto-fill
%%% mode: flyspell
%%% eval: (ispell-change-dictionary "en_US")
%%% TeX-master: "main"
%%% End:
