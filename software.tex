\chapter{Software}
Die Software, die verwendet wird, teilt sich in (?) Teile auf:
\begin{itemize}
\item Auslesen der Sensoren, Aufbereiten der Daten und allgemeine Steuerung (main.sh)
\item Steuern des Displays
\item Endauswertung
\item Webinterface
\item sonstiges
\end{itemize}

\section{main.sh}
\label{sec:main.sh}

Das wichtigste Programm ist das Bash-Script \textit{main.sh}. Mithilfe eines Bash-Scriptes können Programme automatisiert gestartet und ihre Ausgaben ausgewertet werden.
\codeline{main.sh}{bash}{1}
Die Datei beginnt mit einem \textit{Shebang} (auch \textit{Magic Line}) genannt. Diese Zeile sagt dem Betriebssystem, womit die Datei ausgeführt werden soll.
\code{main.sh}{bash}{3}{14}
Die folgenden Zeilen geben allgemeine Einstellungen an und definieren später gebrauchte Variablen. Man kann den Pfad zum Webserver, auf dem das Webinterface liegt, angeben. In Zeile 7 und 8 werden die Zugangsdaten für Pushbullet aus einer anderen Datei ausgelesen. 
